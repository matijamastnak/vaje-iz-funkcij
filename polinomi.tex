\chapter{Polinomi}
\label{cha:polinomi}

\section{Pregled snovi}
\label{sec:polinomi-pregled-snovi}

Foo.

\section{Vaje}
\label{sec:polinomi-funkcije-vaje}

%%%%%%%%%%%%%%%%%%%%%%%%%%%%%%%%%%%%%%%%%%%%%%%%%%%%%%%%%%%%%%%%%%%%%%
% Odpremo datoteko, v katero se bodo zapisali odgovori za
% to poglavje.

% TO ZAENKRAT NE DELUJE
\Opensolutionfile{odgovor}[odgovori-polinomi]

\begin{vaja}
Zgled risanja polinomov:
Narišite graf polinoma $p(x)=x^3-3x+2$.
Potrebujemo:
\begin{itemize}
\item začetno vrednost
\item ničle polinoma
\item kako se funkcija obnaša v neskončnosti
\item ekstreme funkcije
\end{itemize}

Začetna vrednost je enaka $p(0)=2$
Da dobimo ničle, polinom najprej razstavimo s pomočjo Hornerjevega algoritma in dobimo 3 ničle:
$x_{1,2}=1$ (soda) in $x_3=-2$ (liha). Pri sodi ničli se funkcija samo dotakne $x-osi$ in je ne seka, pri lihi pa
funkcija seka $x-os$.
Nato še pogledamo, kako se funkcija obnaša na robovih definicijskega območja. Vidimo, da pri $+\infty$ se graf bliža $+\intfy$,
pri $-\intfy$ pa $-\intfy$.
Zanimajo nas še lokalni ekstremi funkcije, zato izračunamo odvod funkcije: $p\prime(x)=3x^2-3$. Zanima nas, kdaj je enak $0$. Dobimo enačbo $3x^2-3=0$ in dobimo rešitvi $x_1=1$ in $x_2=-1$. Izračunamo še vrednost funkcije v teh točkah in narišemo graf $p(x)$.
% graf p(x)
\end{vaja}

\begin{vaja}
1. naloga iz risanja polinomov:
Narišite graf polinoma $p(x)=-6x^3+21x^2-21x+6$. Pokažite, da v točki z absciso $\frac{3}{2}$ graf polinoma 
seka simetralo lihih kvadrantov. Zapišite še preostali presečišči grafa polinoma s simetralo lihih kvadrantov.



  \begin{odgovor}
1. naloga: %128.naloga

  \end{odgovor}

\end{vaja}



\begin{vaja}

Zgled bisekcije:
Poiščimo iracionalno ničlo $x_0$ polinoma $p(x)=x^5+2x-1$ na desetinko natančno.

Najprej izberemo začetni interval $[a,b]$, da je $p(a)p(b)<0$. Ker je $p(0)=-1<0$ in $p(1)=2>0$, 
vemo, da je ničla nekje na intervalu $[0,1] \Rightarrow$ \\
ničla bo $0,...$ Razpolovna točka tega intervala je $0,5$. \\
$p(0,5)\doteq 0,03>0$, torej je ničla na intervalu $[0;0,5]$ \Rightarrow \\

ničla bi $0,0...$ ali $0,1...$ ali $0,2...$ ali $0,3...$ ali $0,4...$ Razpolovna točka tega intervala je $0,25$.

$p(0,25)\doteq-0,5<0$, torej je ničla na intervalu $[0,25;0,5]$ \Rightarrow \\
ničla bo $0,2...$ ali $0,3...$ ali $0,4...$ Razpolovna točka tega intervala je $0,375$.

$p(0,375)\doteq -0,24<0$, torej je ničla na intervalu $[0,375;0,5]$ \Rightarrow \\
ničla bo $0,3...$ ali $0,4...$ Razpolovna točka tega intervala je $0,4375$.

$p(0,4375)\doteq -0,11<0$, torej je ničla na intervalu $[0,4375;0,5]$ \Rightarrow \\
ničla bo $0,4...$ (prvo decimalno mesto je znano). Razpolovna točka je $0,46875$.

$p(0,46875)\doteq -0,04<0$, torej je ničla na intervalu $[0,46875;0,5]$ \Rightarrow \\
ničla bo $0,46...$ ali $0,47...$ ali $0,48...$ ali $0,49$...

Bisekcijo bi lahko še nadaljevali in bi dobili še naslednja decimalna mesta.
Iracionalna ničla danega polinoma je med $0,47$ in $0,50$.



\end{vaja}

\begin{vaja}
Dan je polinom $p(x)=4x^3-2x^2-x+6$. Z bisekcijo na štiri mesta natančno poiščite ničlo
polinoma $p$ na intervalu $[-2,-1]$.


\begin{odgovor}
Približna vrednost znaša $x\doteq -1,063$.
\end{odgovor}

\end{vaja}








%%%%%%%%%%%%%%%%%%%%%%%%%%%%%%%%%%%%%%%%%%%%%%%%%%%%%%%%%%%%%%%%%%%%%%
% Odgovori

\section{Odgovori}
\label{sec:polinomi-odgovori}

\Closesolutionfile{odgovori-polinomi}

\begin{preodgovor}\preodgovorparams 
 Najprej preverimo, če lahko polinom razstavimo. V tem primeru vidimo, da lahko izpostavimo: $x(x^2-9) - 2(x^2-9)$. Nato še izpostavimo $x^2-9$ in ga razstavimo po razliki kvadratov ter dobimo razcepljen polinom:
$p(x) = (x-2)(x-3)(x+3)$. Zdaj lahko preberemo ničle, ki so: $x_1=2, x_2=3 in x_3=-3$.
  
\end{preodgovor}


%\Readsolutionfile{odgovori-polinomi}



%%% Local Variables:
%%% mode: latex
%%% TeX-master: "vaje"
%%% End:
