\chapter{Polinomi}
\label{cha:polinomi}

\section{Pregled snovi}
\label{sec:polinomi-pregled-snovi}

Foo.

\section{Vaje}
\label{sec:polinomi-funkcije-vaje}

%%%%%%%%%%%%%%%%%%%%%%%%%%%%%%%%%%%%%%%%%%%%%%%%%%%%%%%%%%%%%%%%%%%%%%
% Odpremo datoteko, v katero se bodo zapisali odgovori za
% to poglavje.

% TO ZAENKRAT NE DELUJE
\Opensolutionfile{odgovor}[odgovori-polinomi]

\begin{vaja}
Z razcepom izračunaj ničle polinoma (tudi nerealne):
$x^3-2x^2-9x+18$.

  \begin{odgovor}
 Najprej preverimo, če lahko polinom razstavimo. V tem primeru vidimo, da lahko izpostavimo: $x(x^2-9) - 2(x^2-9)$. Nato še izpostavimo $x^2-9$ in ga razstavimo po razliki kvadratov ter dobimo razcepljen polinom:
$p(x) = (x-2)(x-3)(x+3)$. Zdaj lahko preberemo ničle, ki so: $x_1=2, x_2=3 in x_3=-3$.
  \end{odgovor}
\end{vaja}


%%%%%%%%%%%%%%%%%%%%%%%%%%%%%%%%%%%%%%%%%%%%%%%%%%%%%%%%%%%%%%%%%%%%%%
% Odgovori

\section{Odgovori}
\label{sec:polinomi-odgovori}

\Closesolutionfile{odgovori-polinomi}

\begin{preodgovor}\preodgovorparams 
    Grozna rešitev.
  
\end{preodgovor}


%\Readsolutionfile{odgovori-polinomi}



%%% Local Variables:
%%% mode: latex
%%% TeX-master: "vaje"
%%% End:
