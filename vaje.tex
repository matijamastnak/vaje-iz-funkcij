\documentclass{book}

%%%%%%%%%%%%%%%%%%%%%%%%%%%%%%%%%%%%%%%%%%%%%%%%%%%%%%%%%%%%%%%%%%%%%%
% PAKETI

\usepackage[T1]{fontenc}
\usepackage[utf8]{inputenc}
\usepackage[slovene]{babel}

% Matematika
\usepackage{amsmath}
\usepackage{amssymb}
\usepackage{amsthm}

% Risanje slik
\usepackage{tikz}

% Vaje z rešitvami
\usepackage{answers}

% Naprednejši ukazi
\usepackage{xparse}

% Izbor pisave
\usepackage{mathpazo}
\usepackage[scaled=0.95]{helvet}
\usepackage{courier}
\linespread{1.05} % Pisava Palatino je boljša, če povečamo presledek med vrsticami.


%%%%%%%%%%%%%%%%%%%%%%%%%%%%%%%%%%%%%%%%%%%%%%%%%%%%%%%%%%%%%%%%%%%%%%
% NASLOV, AVTORJI
\title{Vaje iz funkcij}

\author{%
Andrej Bauer \and
Micka Kovačeva
}

%%%%%%%%%%%%%%%%%%%%%%%%%%%%%%%%%%%%%%%%%%%%%%%%%%%%%%%%%%%%%%%%%%%%%%
% OKOLJA ZA IZREKE, DEFINICIJE, ...


%%%%%%%%%%%%%%%%%%%%%%%%%%%%%%%%%%%%%%%%%%%%%%%%%%%%%%%%%%%%%%%%%%%%%%
% Okolje za vaje in rešitve

{\theoremstyle{definition}
\newtheorem{vaja}{Vaja}[chapter]
%\newtheorem{preodgovor}{Odgovor}
}

\Newassociation{odgovor}{Odg}{odgovor}
\renewcommand{\Odglabel}[1]{\textbf{Odgovor #1}}

%%%%%%%%%%%%%%%%%%%%%%%%%%%%%%%%%%%%%%%%%%%%%%%%%%%%%%%%%%%%%%%%%%%%%%
% MAKROJI

% !TeX root = vaje.tex
%%%%%%%%%%%%%%%%%%%%%%%%%%%%%%%%%%%%%%%%%%%%%%%%%%%%%%%%%%%%%%%%%%%%%%
% Množice

% Makro za množice je \set.
% Podamo mu lahko en izbirni argument v oglatih oklepajih []
% in enega ali dva obvezna argumenta v zavitih oklepajih {}.
% Izbirni argument je velikost zavitih oklepajev v zapisu množice.
% Dan je kot število od 0 do 4.
% Če ga ne podamo, se velikost zavitih oklepajev samodejno prilagodi vsebini.
% Če podamo samo en obvezni argument, se množica zapiše kot zaporedje elementov v zavitih oklepajih.
% Če podamo dva obvezna argumenta, se ta dva izpišeta, ločena z navpično črto in obdana z zavitimi oklepaji.
% Primer:
% \set{1, 2, 3}  izpiše  {1, 2, 3}.
% \set{x \in \RR}{x \geq 0}  izpiše  {x ∈ ℝ | x ≥ 0}.

\newcommand{\sizedescriptor}[2]
{
\ifthenelse{\equal{#1}{0}}{}{
\ifthenelse{\equal{#1}{1}}{\big}{
\ifthenelse{\equal{#1}{2}}{\Big}{
\ifthenelse{\equal{#1}{3}}{\bigg}{
\ifthenelse{\equal{#1}{4}}{\Bigg}{
#2}}}}}
}

\NewDocumentCommand{\set}
{O{auto} m G{\empty}}
{\sizedescriptor{#1}{\left}\{ {#2} \ifthenelse{\equal{#3}{}}{}{ \; \sizedescriptor{#1}{\middle}| \; {#3}} \sizedescriptor{#1}{\right}\}}


%%%%%%%%%%%%%%%%%%%%%%%%%%%%%%%%%%%%%%%%%%%%%%%%%%%%%%%%%%%%%%%%%%%%%%
% Številske množice

\newcommand{\NN}{\mathbb{N}}     % naravna števila
\newcommand{\NNz}{\mathbb{N}_0}  % naravna števila z ničlo
\newcommand{\ZZ}{\mathbb{Z}}     % cela števila
\newcommand{\QQ}{\mathbb{Q}}     % racionalna števila
\newcommand{\RR}{\mathbb{R}}     % realna števila


%%% Local Variables:
%%% mode: latex
%%% TeX-master: "vaje"
%%% End:


\begin{document}

\maketitle

%%%%%%%%%%%%%%%%%%%%%%%%%%%%%%%%%%%%%%%%%%%%%%%%%%%%%%%%%%%%%%%%%%%%%%
% KAZALO

\setcounter{tocdepth}{0} % Prikaži samo poglavja (nastavi na 1 za razdelke)

\tableofcontents

%%%%%%%%%%%%%%%%%%%%%%%%%%%%%%%%%%%%%%%%%%%%%%%%%%%%%%%%%%%%%%%%%%%%%%
% VSEBINA

\chapter{Uvod}
\label{cha:uvod}

Tu bo en lep uvod.

%%% Local Variables:
%%% mode: latex
%%% TeX-master: "vaje"
%%% End:

\chapter{Polinomi}
\label{cha:polinomi}

\section{Pregled snovi}
\label{sec:polinomi-pregled-snovi}

Foo.

\section{Vaje}
\label{sec:polinomi-funkcije-vaje}

%%%%%%%%%%%%%%%%%%%%%%%%%%%%%%%%%%%%%%%%%%%%%%%%%%%%%%%%%%%%%%%%%%%%%%
% Odpremo datoteko, v katero se bodo zapisali odgovori za
% to poglavje.

% TO ZAENKRAT NE DELUJE
\Opensolutionfile{odgovor}[odgovori-polinomi]

\begin{vaja}
Zgled risanja polinomov:
Narišite graf polinoma $p(x)=x^3-3x+2$.
Potrebujemo:
\begin{itemize}
\item začetno vrednost
\item ničle polinoma
\item kako se funkcija obnaša v neskončnosti
\item ekstreme funkcije
\end{itemize}

Začetna vrednost je enaka $p(0)=2$
Da dobimo ničle, polinom najprej razstavimo s pomočjo Hornerjevega algoritma in dobimo 3 ničle:
$x_{1,2}=1$ (soda) in $x_3=-2$ (liha). Pri sodi ničli se funkcija samo dotakne $x-osi$ in je ne seka, pri lihi pa
funkcija seka $x-os$.
Nato še pogledamo, kako se funkcija obnaša na robovih definicijskega območja. Vidimo, da pri $+\infty$ se graf bliža $+\infty$,
pri $-\infty$ pa $-\infty$.
Zanimajo nas še lokalni ekstremi funkcije, zato izračunamo odvod funkcije: $p\prime(x)=3x^2-3$. Zanima nas, kdaj je enak $0$. Dobimo enačbo $3x^2-3=0$ in dobimo rešitvi $x_1=1$ in $x_2=-1$. Izračunamo še vrednost funkcije v teh točkah in narišemo graf $p(x)$.
% graf p(x)
\end{vaja}

\begin{vaja}
1. naloga iz risanja polinomov:
Narišite graf polinoma $p(x)=-6x^3+21x^2-21x+6$. Pokažite, da v točki z absciso $\frac{3}{2}$ graf polinoma 
seka simetralo lihih kvadrantov. Zapišite še preostali presečišči grafa polinoma s simetralo lihih kvadrantov.



  \begin{odgovor}
1. naloga: %128.naloga

  \end{odgovor}

\end{vaja}



\begin{vaja}

Zgled bisekcije:
Poiščimo iracionalno ničlo $x_0$ polinoma $p(x)=x^5+2x-1$ na desetinko natančno.

Najprej izberemo začetni interval $[a,b]$, da je $p(a)p(b)<0$. Ker je $p(0)=-1<0$ in $p(1)=2>0$, 
vemo, da je ničla nekje na intervalu $[0,1] \Rightarrow$ \\
ničla bo $0,...$ Razpolovna točka tega intervala je $0,5$. \\
$p(0,5)\doteq 0,03>0$, torej je ničla na intervalu $[0;0,5] \Rightarrow$ \\

ničla bi $0,0...$ ali $0,1...$ ali $0,2...$ ali $0,3...$ ali $0,4...$ Razpolovna točka tega intervala je $0,25$.

$p(0,25)\doteq-0,5<0$, torej je ničla na intervalu $[0,25;0,5] \Rightarrow$ \\
ničla bo $0,2...$ ali $0,3...$ ali $0,4...$ Razpolovna točka tega intervala je $0,375$.

$p(0,375)\doteq -0,24<0$, torej je ničla na intervalu $[0,375;0,5] \Rightarrow$ \\
ničla bo $0,3...$ ali $0,4...$ Razpolovna točka tega intervala je $0,4375$.

$p(0,4375)\doteq -0,11<0$, torej je ničla na intervalu $[0,4375;0,5] \Rightarrow$ \\
ničla bo $0,4...$ (prvo decimalno mesto je znano). Razpolovna točka je $0,46875$.

$p(0,46875)\doteq -0,04<0$, torej je ničla na intervalu $[0,46875;0,5] \Rightarrow$ \\
ničla bo $0,46...$ ali $0,47...$ ali $0,48...$ ali $0,49$...

Bisekcijo bi lahko še nadaljevali in bi dobili še naslednja decimalna mesta.
Iracionalna ničla danega polinoma je med $0,47$ in $0,50$.



\end{vaja}

\begin{vaja}
Dan je polinom $p(x)=4x^3-2x^2-x+6$. Z bisekcijo na štiri mesta natančno poiščite ničlo
polinoma $p$ na intervalu $[-2,-1]$.


\begin{odgovor}
Približna vrednost znaša $x\doteq -1,063$.
\end{odgovor}

\end{vaja}








%%%%%%%%%%%%%%%%%%%%%%%%%%%%%%%%%%%%%%%%%%%%%%%%%%%%%%%%%%%%%%%%%%%%%%
% Odgovori

\section{Odgovori}
\label{sec:polinomi-odgovori}

\Closesolutionfile{odgovori-polinomi}

\begin{preodgovor}\preodgovorparams 
 Najprej preverimo, če lahko polinom razstavimo. V tem primeru vidimo, da lahko izpostavimo: $x(x^2-9) - 2(x^2-9)$. Nato še izpostavimo $x^2-9$ in ga razstavimo po razliki kvadratov ter dobimo razcepljen polinom:
$p(x) = (x-2)(x-3)(x+3)$. Zdaj lahko preberemo ničle, ki so: $x_1=2, x_2=3 in x_3=-3$.
  
\end{preodgovor}


%\Readsolutionfile{odgovori-polinomi}



%%% Local Variables:
%%% mode: latex
%%% TeX-master: "vaje"
%%% End:

% !Tex root = vaje.tex
\chapter{Eksponentna in logaritemska funkcija}
\label{cha:exp-log}

\section{Pregled snovi}
\label{sec:exp-log-pregled-snovi}

Pregled snovi.

\section{Vaje}
\label{sec:exp-log-vaje}

%%%%%%%%%%%%%%%%%%%%%%%%%%%%%%%%%%%%%%%%%%%%%%%%%%%%%%%%%%%%%%%%%%%%%%
% Odpremo datoteko, v katero se bodo zapisali odgovori za
% to poglavje.

% Določimo ime datoteke, v katero se bodo pisali odgovori.
% Vsako poglavje mora imeti svojo datoteko.
\def\datotekaOdgovori{odgovori-explog}

% Odpremo datoteko z odgovori.
\Opensolutionfile{odgovor}[\datotekaOdgovori]

%%%%%%%%%%%%%%%%%%%%%%%%%%%%%%%%%%%%%%%%%%%%%%%%%%%%%%%%%%%%%%%%%%%%%%
% VAJE
%
% Sem vstavimo vaje s pomočjo okolja "vaja". Odgovor napišemo v vajo,
% v okolje "odgovor".

\begin{vaja}
  Izračunajte $2^4$.

  \begin{odgovor}
    $2^4 = 4^2$.
  \end{odgovor}
\end{vaja}

\begin{vaja}
  Še ena vaja.

  \begin{odgovor}
    Rešitev bi bila tu.
  \end{odgovor}
\end{vaja}

%%%%%%%%%%%%%%%%%%%%%%%%%%%%%%%%%%%%%%%%%%%%%%%%%%%%%%%%%%%%%%%%%%%%%%
% Treba je zapredi datoteko z odgovori

\Closesolutionfile{odgovor}

%%%%%%%%%%%%%%%%%%%%%%%%%%%%%%%%%%%%%%%%%%%%%%%%%%%%%%%%%%%%%%%%%%%%%%
% Odgovori

\section{Odgovori}
\label{sec:explog-odgovori}

% Vključimo odgovore.
\input{\datotekaOdgovori}


%%% Local Variables:
%%% mode: latex
%%% TeX-master: "vaje"
%%% End:

% !TeX root = vaje.tex

\chapter{Kotne funkcije}
\label{cha:sin-cos}

\section{Pregled snovi}
\label{sec:sin-cos-pregled-snovi}

Pregled snovi.

\section{Vaje}
\label{sec:sin-cos-vaje}

%%%%%%%%%%%%%%%%%%%%%%%%%%%%%%%%%%%%%%%%%%%%%%%%%%%%%%%%%%%%%%%%%%%%%%
% Odpremo datoteko, v katero se bodo zapisali odgovori za
% to poglavje.

% Določimo ime datoteke, v katero se bodo pisali odgovori.
% Vsako poglavje mora imeti svojo datoteko.
\def\datotekaOdgovori{odgovori-sincos}

% Odpremo datoteko z odgovori.
\Opensolutionfile{odgovor}[\datotekaOdgovori]

%%%%%%%%%%%%%%%%%%%%%%%%%%%%%%%%%%%%%%%%%%%%%%%%%%%%%%%%%%%%%%%%%%%%%%
% VAJE
%
% Sem vstavimo vaje s pomočjo okolja "vaja". Odgovor napišemo v vajo,
% v okolje "odgovor".

\begin{vaja}
  Izračunajte $\sin(1928398213 \pi)$.

  \begin{odgovor}
    $0$.
  \end{odgovor}
\end{vaja}

\begin{vaja}
  Še ena vaja.

  \begin{odgovor}
    Rešitev bi bila tu.
  \end{odgovor}
\end{vaja}

%%%%%%%%%%%%%%%%%%%%%%%%%%%%%%%%%%%%%%%%%%%%%%%%%%%%%%%%%%%%%%%%%%%%%%
% Treba je zapredi datoteko z odgovori

\Closesolutionfile{odgovor}

%%%%%%%%%%%%%%%%%%%%%%%%%%%%%%%%%%%%%%%%%%%%%%%%%%%%%%%%%%%%%%%%%%%%%%
% Odgovori

\section{Odgovori}
\label{sec:sincos-odgovori}

% Vključimo odgovore.
\input{\datotekaOdgovori}


%%% Local Variables:
%%% mode: latex
%%% TeX-master: "vaje"
%%% End:



\end{document}
